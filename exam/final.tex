\documentclass[main.tex]{subfiles}
\begin{document}

\begin{enumerate}
\item [1.] \textbf{Q.} Solve differential equation:

$$
\mathrm{dU}(t) / \mathrm{dt}=\left[\begin{array}{ll}
0 & 1 \\
1 & 0
\end{array}\right] \mathrm{U}(t) ; \quad \mathrm{U}(0)=\left[\begin{array}{l}
4 \\
2
\end{array}\right]
$$

solve for $\mathrm{U}(\mathrm{t})$; and evaluate $\mathrm{U}(1)$. \textbf{A.}

$$
\begin{aligned}
\frac{d\bm{u}}{dt} & = A\bm{u}\\
\frac{d}{dt} \left[\begin{array}{l}
y \\
z 
\end{array}\right] & = \left[\begin{array}{ll}
0 & 1\\
1 & 0
\end{array}\right] \left[\begin{array}{l}
y \\
z 
\end{array}\right]\\
\frac{dy}{dt} &= z\\
\frac{dz}{dt} &= y\\
\frac{d}{dt}(y+z) &= z+y\\
\frac{d}{dt}(y-z) &= -(y-z)\\
\end{aligned}
$$

The matrix $\left[\begin{array}{ll} 0 & 1 \\ 1 & 0 \end{array}\right]$ has eigen values 1 and -1, and eigen vectors (1,1) and (1,-1).

$$
\begin{aligned}
\bm{u}_1 & = e^{\lambda_1 t}\bm{x}_1 = e^t \left[\begin{array}{l}
1\\
1
\end{array}\right] \\
\bm{u}_2 & = e^{\lambda_2 t}\bm{x}_2 = e^-t \left[\begin{array}{l}
1\\
-1
\end{array}\right] \\
\bm{u}(t) & = C e^t \left[\begin{array}{l} 1 \\ 1 \end{array}\right] 
+ D e^{-t} \left[\begin{array}{l} 1 \\ -1 \end{array}\right] \\
\because \mathrm{U}(0) & =\left[\begin{array}{l}
4 \\
2
\end{array}\right] \\
\bm{u}(t) & = 3 e^t \left[\begin{array}{l} 1 \\ 1 \end{array}\right] 
+ e^{-t} \left[\begin{array}{l} 1 \\ -1 \end{array}\right]\\
\bm{u}(1) & = 3 e^1 \left[\begin{array}{l} 1 \\ 1 \end{array}\right] 
+ e^{-1} \left[\begin{array}{l} 1 \\ -1 \end{array}\right] 
= \left[\begin{array}{l} 3 e + e^{-1} \\ 3 e - e^{-1} \end{array}\right] \\
\end{aligned}
$$


\item[2.] 
    \begin{enumerate}
    \item [a.] \textbf{Q.} Find a basis for the space of all vectors $\left(x_{1}, x_{2}, x_{3}, x_{4}\right)$ that are orthogonal (perpendicular) to both of these vectors:
    $$
    \left[\begin{array}{l}
    1 \\
    2 \\
    3 \\
    1
    \end{array}\right] \quad\left[\begin{array}{l}
    0 \\
    0 \\
    1 \\
    2
    \end{array}\right]
    $$
    \textbf{A.}

    The dot product of two orthogonal vectors is zero.

    $$
    \begin{aligned}
    x_1 + 2x_2 + 3x_3 + x_4 &= 0\\
    x_3 + 2x_4 &= 0\\
    x_3 &= -2x_4\\
    x_1 + 2x_2 + 3(-2x_4) + x_4 &= 0\\
    x_1 &= -2x_2 + 5x_4\\
    \left[\begin{array}{l}
    x_1 \\
    x_2 \\
    x_3 \\
    x_4
    \end{array}\right] & = \left[\begin{array}{l}
    -2x_2 + 5x_4 \\
    x_2 \\
    -2x_4 \\
    x_4
    \end{array}\right] = x_2\left[\begin{array}{l}
    -2 \\
    1 \\
    0 \\
    0
    \end{array}\right] + x_4\left[\begin{array}{l}
    5 \\
    0 \\
    -2 \\
    1
    \end{array}\right]
    \end{aligned}
    $$

    The vector space is spanned by (-2,1,0,0) and (5,0,-2,1). If these vectors are linearly independent they form the basis.

    $$
    \begin{aligned}
    x(-2,1,0,0) + y(5,0,-2,1) & = (0,0,0,0)\\
    -2x + 5y & = 0\\
    x &= 0\\
    -2y &= 0 \\
    y & = 0
    \end{aligned}
    $$

    The basis of the vector space perpendicular to $(1,2,3,1)$ and $(0,0,1,2)$ is:
    
    $$
    \left(\left[\begin{array}{c}
    -2 \\
    1 \\
    0 \\
    0
    \end{array}\right],\left[\begin{array}{c}
    5 \\
    0 \\
    -2 \\
    1
    \end{array}\right]\right)
    $$
    
    \item [b.] \textbf{Q.} If $u, v, w$ are three nonzero vectors in $R^{7}$, what are the possible dimensions of the subspace they span? 
    \textbf{A.} When $\mathrm{u}, \mathrm{v}$ and $\mathrm{w}$ are linearly independent the dimension of spanned subspace is three (dimension of vector space is equal to the number of linearly independent vectors in its spanning subset). When two vectors out of $u, v$ and $w$ are linearly independent the third one can be written as a linear function of the other two vectors and the dimension of spanned subspace is two. when any two vectors can be written as a linear function of the third vector the dimension of spanned subspace is one. The possible dimensions are 1, 2 and 3.

    \end{enumerate}

\item[3.] Let $T: \mathrm{R}^{2} \rightarrow \mathrm{R}^{2}$ be the linear transformation satisfying
    $$
    T\left(\left[\begin{array}{l}
    1 \\
    0
    \end{array}\right]\right)=\left[\begin{array}{r}
    -4 \\
    3
    \end{array}\right] \quad \text { and } \quad T\left(\left[\begin{array}{c}
    1 \\
    1
    \end{array}\right]\right)=\left[\begin{array}{c}
    -10 \\
    8
    \end{array}\right]
    $$
    \begin{enumerate}
        \item [a.] \textbf{Q.} Find $T\left(\left[\begin{array}{l}0 \\ 1\end{array}\right]\right)$ 
        \textbf{A.}

        $$
        \begin{aligned}
        T(v+w) & = T(v) + T(w)\\
        T\left(\left[\begin{array}{l}1 \\ 1\end{array}\right]\right) 
        & = T\left(\left[\begin{array}{l}1 \\ 0\end{array}\right]\right)+T\left(\left[\begin{array}{l}0 \\ 1\end{array}\right]\right)\\
        \left[\begin{array}{c}
        -10 \\
        8
        \end{array}\right] &= \left[\begin{array}{c}
        -4 \\
        3
        \end{array}\right] + T\left(\left[\begin{array}{l}
        0 \\
        1
        \end{array}\right]\right) \\
        T\left(\left[\begin{array}{l}
        0 \\
        1
        \end{array}\right]\right) &= \left[\begin{array}{c}
        -6 \\
        5
        \end{array}\right]
        \end{aligned}
        $$
        
        \item [b.] \textbf{Q.} What is the matrix $A$ expressing $T$ in terms of the standard basis vectors $\left[\begin{array}{l}1 \\ 0\end{array}\right]$ and $\left[\begin{array}{l}0 \\ 1\end{array}\right] \? $ (The same basis is used for the input and the output.) 
        \textbf{A.}

        $$
        \begin{aligned}
        T\left(\left[\begin{array}{l}
        1 \\
        0
        \end{array}\right]\right)&=\left[\begin{array}{c}
        -4 \\
        3
        \end{array}\right]=-4\left[\begin{array}{l}
        1 \\
        0
        \end{array}\right]+3\left[\begin{array}{l}
        0 \\
        1
        \end{array}\right] \\
        T\left(\left[\begin{array}{l}
        0 \\
        1
        \end{array}\right]\right)&=\left[\begin{array}{c}
        -6 \\
        5
        \end{array}\right]=-6\left[\begin{array}{l}
        1 \\
        0
        \end{array}\right]+5\left[\begin{array}{l}
        0 \\
        1
        \end{array}\right]\\
        A&=\left[\begin{array}{cc}
        -4 & -6 \\
        3 & 5
        \end{array}\right] \\
        T\left(\left[\begin{array}{l}
        x \\
        y
        \end{array}\right]\right)&=\left[\begin{array}{cc}
        -4 & -6 \\
        3 & 5
        \end{array}\right]\left[\begin{array}{l}
        x \\
        y
        \end{array}\right]
        \end{aligned}
        $$
        
        \item [c.] \textbf{Q.} What is the matrix $B$ expressing $T$ in terms of the basis consisting of eigenvectors of $A$? (The same basis is used for the input and output.) (There are two possible correct answers, depending on what order you pick the eigenvectors.) 
        \textbf{A.}

        $$
        \begin{aligned}
        |A-\lambda I|&=0 \\
        \left|\begin{array}{cc}
        -4-\lambda & -6 \\
        3 & 5-\lambda
        \end{array}\right|&=0\\
        (-4-\lambda)(5-\lambda) + 18 &= 0\\
        \lambda^2 - \lambda - 2 & = 0\\
        \lambda_{1}, \lambda_{2} &= \frac{1 \pm \sqrt{1-4(-2)}}{2}\\
        &= \frac{1 \pm \sqrt{9}}{2}\\
        \lambda_1 & = -1\\
        \lambda_2 & = 2\\
        B &=\left[\begin{array}{cc}
        -1 & 0 \\
        0 & 2
        \end{array}\right]
        \end{aligned}
        $$
        
    \end{enumerate}

\item[4.] Let $A=\left[\begin{array}{rr}4 & 1 \\ -1 & 4\end{array}\right]$.
    \begin{enumerate}
    \item [1a.] \textbf{Q.} Find the eigenvalues of $A$. 
    \textbf{A.}

    $$
    \begin{aligned}
    |A-\lambda I| &= 0\\
    \left|\begin{array}{cc}
    4-\lambda & 1 \\
    -1 & 4-\lambda
    \end{array}\right| &= 0 \\
    (4-\lambda)(4-\lambda)+1 &= 0 \\
    16-8 \lambda+\lambda^{2}+1 &= 0 \\
    \lambda^{2}-8 \lambda+17 &= 0 \\
    \lambda_1,\lambda_2 &= \frac{8 \pm \sqrt{64-4(17)}}{2}\\
    & = \frac{8 \pm \sqrt{-4}}{2}\\
    & = 4\pm i
    \end{aligned}
    $$

    \end{enumerate}

\item[5.] Each independent question refers to the matrix $A=\left[\begin{array}{cc}4 & 1 \\ d & -4\end{array}\right]$. In each case, find the value of $d$ that makes the statement true (and show your work!).
\begin{enumerate}
    \item [4a.] \textbf{Q.} Give a value for $d$ such that $\left[\begin{array}{l}5 \\ 1\end{array}\right]$ is an eigenvector of $A$. 
    \textbf{A.}

    $$
    \begin{aligned}
    \operatorname{trace}(A)&=4-4\\
    & =0\\
    & = \lambda_1 + \lambda_2\\
    \lambda_1 &= -\lambda_2\\
    (A-\lambda I)\bm{x} & = \bm{0}\\
    \left[\begin{array}{cc}4-\lambda & 1 \\ d & -4-\lambda\end{array}\right]
    \left[\begin{array}{l}5 \\ 1\end{array}\right] 
    & = \left[\begin{array}{l}0 \\ 0\end{array}\right]\\
    21-5\lambda &= 0\\
    \lambda &= \frac{21}{5}\\
    5d - 4 -\lambda & = 0\\
    d &= \frac{41}{25}
    \end{aligned}
    $$
    
\end{enumerate}

\item[6.] We are given two vectors $a$ and $b$ in $\mathbb{R}^{4}$ :
$$
a=\left[\begin{array}{l}
2 \\
5 \\
2 \\
4
\end{array}\right] \quad b=\left[\begin{array}{l}
1 \\
2 \\
1 \\
0
\end{array}\right]
$$
    \begin{enumerate}
    
    \item [a.] \textbf{Q.} Find the projection $p$ of the vector $b$ onto the line through $a$. Check (!) that the error $e=b-p$ is perpendicular to (what?)
    \textbf{A.}

    $$
    \begin{aligned}
    \bm{a}^T \bm{b} &= \left[\begin{array}{llll} 2 & 5 & 2 & 4 \end{array}\right]
    \left[\begin{array}{l}
    1 \\
    2 \\
    1 \\
    0
    \end{array}\right] = 14\\
    \bm{a}^T \bm{a} &= \left[\begin{array}{llll} 2 & 5 & 2 & 4 \end{array}\right]
    \left[\begin{array}{l}
    2 \\
    5 \\
    2 \\
    4
    \end{array}\right] = 49\\
    \boldsymbol{p}&=\frac{\boldsymbol{a}^{T} \boldsymbol{b}}{\boldsymbol{a}^{T} \boldsymbol{a}} \boldsymbol{a}\\
    & = \frac{2}{7} \left[\begin{array}{l}
    2 \\
    5 \\
    2 \\
    4
    \end{array}\right] \\
    & = \frac{1}{7}\left[\begin{array}{l}
    4 \\
    10 \\
    4 \\
    8
    \end{array}\right] \\
    \bm{e} &= \bm{b} - \bm{p} \\
    & = \left[\begin{array}{l}
    1 \\
    2 \\
    1 \\
    0
    \end{array}\right] - \frac{1}{7}\left[\begin{array}{l}
    4 \\
    10 \\
    4 \\
    8
    \end{array}\right] \\
    & = \frac{1}{7} \left[\begin{array}{l}
    3 \\
    4 \\
    3 \\
    -8
    \end{array}\right] \\
    \bm{e} \cdot \bm{a} & = \frac{1}{7} \left[\begin{array}{l}
    3 \\
    4 \\
    3 \\
    -8
    \end{array}\right] \cdot \left[\begin{array}{l}
    2 \\
    5 \\
    2 \\
    4
    \end{array}\right] \\
    & = \frac{1}{7} \left[6 + 20 + 6 - 32 \right] \\
    & = 0
    \end{aligned}
    $$
    
    \item [b.] \textbf{Q.} The subspace $S$ of all vectors in $\mathbb{R}^{4}$ that are perpendicular to this $a$ is 3-dimensional. Find the projection $q$ of $b$ onto this perpendicular subspace $S$. The numerical answer (it doesn't need a big computation!) is $q=\underline{\hspace{2cm}}$.
    \textbf{A.}

    $$
    \begin{aligned}
    \bm{s} \cdot \bm{a} & = 0\\
    \bm{s} &= s_{1} \hat{x} + s_{2} \hat{y} + s_{3} \hat{z} +s_{4} \hat{p}\\
    & = 5 \hat{x} - 4 \hat{y} + 5 \hat{z}\\
    \bm{s} \cdot \bm{b} &= 5(1)-4(2)+5(1) \\
    & = 5-8+5 \\
    & = 2 \\
    |\bm{s}| & = \sqrt{5^{2}-4^{2}+5^{2}} \\
    & = \sqrt{66} \\
    q &= \frac{\bm{s} \cdot \bm{b}}{|\bm{s}|}\\
    & = \frac{2}{\sqrt{66}}
    \end{aligned}
    $$
    
    \end{enumerate} 

\item[7.] Consider the following operations on the space of quadratic polynomials $f(x)=a x^{2}+b x+c$. Which of them are linear transformations? If they are linear transformations, find their matrices in the basis $1, x, x^{2}$. If they are not linear transformations, explain it using the definition of linear transformation.
    \begin{enumerate}
    \item [a.] \textbf{Q.} $T_{1}(f)=f(x)-f(1)$ 
    \textbf{A.}
    
    $v=$ space of quadratic polynomials $f(x) = ax^{2} + bx + c$, let $f, g \in v$ and $\alpha \in \mathbb{R}$.

    $$
    \begin{aligned}
    T_{1}(f + cg) &= (f+cg)(x)-(f+cg)(1) \\
    &= f(x)+c g(x)-f(1)-c g(1) \\
    &= [f(x)-f(1)]+c[g(x)-g(1)] \\
    &= T_{1}(f) + c T_{1}(g) \\
    \end{aligned}
    $$

    $\therefore T_{1}$ is a linear transformation.

    $$
    \begin{aligned}
    T_{1}(1) &= 1-1\\ 
    & = 0 \cdot 1+0 \cdot x+0 \cdot x^{2}\\
    T_{1}(x) &= x-1 \\
    & = (-1) \cdot 1 + 1 \cdot x + 0 \cdot x^{2}\\
    T_{1}(x^{2}) &= x^{2}-1 \\
    &= (-1) 1 + 0 \cdot x + 1 \cdot x^{2}\\
    \left[T_1\right] &= \left[\begin{array}{ccc}
    0 & -1 & -1 \\
    0 & 1 & 0 \\
    0 & 0 & 1
    \end{array}\right]
    \end{aligned}
    $$
    
    \item [b.] \textbf{Q.} $T_{2}(f)=f(x)-1$ 
    \textbf{A.}

    $$
    \begin{aligned}
    f(x) &= x\\
    g(x) &= x+1\\
    T_{2}(f) &= f(x)- 1 \\
    &=x-1 \\
    T_{2}(g)&= g(x)-1 \\
    &=x+1-1 \\
    &=x \\
    T_{2}(f+g) &= (f+y)(x)-1 \\
    &= f(x)+g(x)-1 \\
    &= (2 x+1)-1 \\
    &= 2x \\
    T_{2}(f+g) &\neq T_{2}(f)+T_{2}(g)
    \end{aligned}
    $$

    $T_{2}$ is not a linear transformation.
    
    \item [c.] \textbf{Q.} $T_{3}(f)=x-f(1)$
    \textbf{A.}

    $$
    \begin{aligned}
    f(x) &= x\\
    g(x) &= x+1\\
    T_{3}(f) &= x-f(1)\\
    &=x-1\\
    T_{3}(g) &= x-g(1)\\
    &=x-(1+1)\\
    &=x-2\\
    T_{3}(f+g) &= x-(2 \cdot 1+1)\\
    &=x-3\\
    T_{3}(f)+T_{3}(g)&=(x-1)+(x-2)\\
    &=2 x-3\\
    T_{3}(f)+T_{3}(g) &\neq T_{3}(f+g)\\
    \end{aligned}
    $$

    $T_{3}$ is not a linear transformation
    
    \item [d.] \textbf{Q.} $T_{4}(f)=x^{2} f(1 / x)$
    \textbf{A.}

    $$
    \begin{aligned}
    T_{4}(f) &= x^{2} \cdot f(1 / x)\\
    T_{4}(g) &= x^{2} \cdot g(1 / x) \\
    T_{4}(f + cg) &= x^{2} \cdot(f+c g)(1 / x) \\
    &=x^{2} \cdot[f(1 / x)+c \cdot g(1 / x)] \\
    &=x^{2} f(1 / x)+c \cdot x^{2} g(1 / x) \\
    &=T_{4}(f)+c \cdot T_{4}(g)
    \end{aligned}
    $$

    $T_4$ is a linear transformation.

    $$
    \begin{aligned}
    T_{4}(1) &= x^{2} \cdot 1\\
    &= 0 \cdot 1+0 \cdot x+1 \cdot x^{2} \\
    T_{4}\left(x^{2}\right) &= x^{2} \cdot \frac{1}{x}\\
    &= 0 \cdot 1+1 \cdot x+0 \cdot x^{2} \\
    T_{4}\left(x^{2}\right) &= x^{2} \cdot \frac{1}{x^{2}}\\
    &= 1 \cdot 1+0 \cdot x+0 \cdot x^{2} \\
    \left[T_{4}\right]&=\left[\begin{array}{lll}
    0 & 0 & 1 \\
    0 & 1 & 0 \\
    1 & 0 & 0
    \end{array}\right]
    \end{aligned}
    $$

    
    \item [e.] \textbf{Q.} The linear transformation $R: \mathbb{R}^{2} \longrightarrow \mathbb{R}^{2}$ is the reflection with respect to the line $x+y=0$. Find the matrix of $R$ in the standard basis of $\mathbb{R}^{2}$.
    \textbf{A.}

    $$
    \begin{aligned}
    R &=\left[\begin{array}{lll}
    0 & 0 & -1 \\
    0 & -1 & 0 \\
    -1 & 0 & 0    
    \end{array}\right]
    \end{aligned}
    $$
    
    \end{enumerate}

\end{enumerate}

\end{document}